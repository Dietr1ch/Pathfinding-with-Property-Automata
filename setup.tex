\usepackage{ucs}
\usepackage{auto-pst-pdf}

% Template setup
% ==============

% Floats
\usepackage{graphicx}
\usepackage{float}
\floatstyle{boxed}
\restylefloat{figure}
\usepackage{subfigure}
\usepackage{color}

% Math packages
\usepackage{amsmath}
\usepackage{amsfonts}
\usepackage{amssymb}

% Closest font to Times New Roman
\usepackage{times}

% To make pretty tables
\usepackage{booktabs}
\usepackage{multirow}

% To avoid underfull errors in the bibliography
\usepackage{etoolbox}
\apptocmd{\sloppy}{\hbadness 10000\relax}{}{}

% To make cites and references
%\usepackage[hidelinks,pdfusetitle,pdfdisplaydoctitle]{hyperref}
\hypersetup{
    hidelinks=true,
    pdfusetitle=true,
    pdfdisplaydoctitle=true
}

\usepackage[notocbib]{apacite}
\usepackage{doi}
\renewcommand{\doitext}{}

% New environments
\newtheorem{definition}{\bf Definition}[chapter]
\newtheorem{property}{Property}[chapter]
\newtheorem{claim}{Claim}[chapter]
\newtheorem{lemma}{\bf Lemma}[chapter]
\newtheorem{proposition}{Proposition}[chapter]
\newtheorem{theorem}{\noindent \bf Theorem}[chapter]
\newtheorem{corollary}{\bf Corollary}[chapter]
\newtheorem{pf}{Proof}[chapter]
\newtheorem{example}{\bf Example}[chapter]
\newtheorem{remark}{Remark}[chapter]



% Additional setup
% ================
\usepackage[left=4cm, top=3cm, bottom=4cm, right=2.5cm]{geometry}


% Defines
% -------
\usepackage{mathtools}
\newcommand{\on}[1]{\operatorname{#1}}
\DeclarePairedDelimiter\set{\{}{\}}
\DeclarePairedDelimiter\abs{|}{|}
\DeclarePairedDelimiter\sizeof{\|}{\|}
\DeclarePairedDelimiter\ceil{\lceil}{\rceil}
\DeclarePairedDelimiter\floor{\lfloor}{\rfloor}

% Number sets
\newcommand{\Booleans}{\mathbb{B}}
\newcommand{\Naturals}{\mathbb{N}}
\newcommand{\Integers}{\mathbb{Z}}
\newcommand{\Rationals}{\mathbb{Q}}
\newcommand{\Reals}{\mathbb{R}}
\newcommand{\Complex}{\mathbb{C}}

\newcommand{\Pred}[1]{{\mathcal{P}(#1)}}
\newcommand{\IRI}{\mathbf{I}}
\newcommand{\Blank}{\mathbf{B}}
\newcommand{\Lit}{\mathbf{L}}
\newcommand{\Gr}{\mathbf{G}}
\newcommand{\LG}{\Gr_{\Sigma}^{\Tau}}
\newcommand{\SWG}{\mathcal{G_W}}
\newcommand{\RDF}{\mathcal{R}}
\newcommand{\A}{\mathcal{A}}

% \newcommand{\Tau}{\mathrm{T}}
\newcommand{\Tau}{\mathrm{T}}

\usepackage{numprint}
\npthousandsep{,}


% Tikz
\usepackage[ruled,vlined,linesnumbered]{algorithm2e}
\usepackage{tikz}
\usetikzlibrary{arrows,arrows.meta,automata,snakes,positioning,shapes,shapes.geometric,shadows,calc}
\tikzset{>={Latex[width=3mm,length=3mm]}}
\tikzset{svertex style/.style={
    draw=#1,
    thick,
    fill=#1!70,
    text=white,
    circle,
    minimum width=0.5cm,
    minimum height=0.5cm,
    font=\footnotesize{},
    outer sep=3pt, % the usage of this option will be clear later on
  },
}

%%  % Times fonts, as required by the OGRS
%%  \usepackage{mathptmx}


%%  \usepackage{subfig}
%%  \usepackage{fancybox}
%%  \usepackage[T1]{fontenc}
%%
%%
%%  \usepackage{pdfpages}
%%
%%  \usepackage{booktabs}
%%
%%  \usepackage{times}
%%  \usepackage{graphicx}
%%  \usepackage{epic}
%%  \usepackage{eepic}
%%  \usepackage{url}
%%  \usepackage{algorithm2e}
%%  \usepackage{rotfloat}
%%  \usepackage{xspace}
%%  \usepackage{amsmath}
%%  \usepackage{amssymb}
%%  \usepackage{use stmaryrd}
%%  \usepackage{cases}
%%
%%  % Tikz
%%  \usepackage{tikz}
%%   \usetikzlibrary{trees}
%%   \usetikzlibrary{shapes}
%%   \usetikzlibrary{shapes.geometric,shapes}
%%   \usetikzlibrary{positioning}
%%   \usetikzlibrary{calc}
%%   \usetikzlibrary{arrows}
%%   \usetikzlibrary{shadows}
%%  \usepackage{tikz-qtree,tikz-qtree-compat}
